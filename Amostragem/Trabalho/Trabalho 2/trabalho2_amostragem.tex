\documentclass[12pt,a4paper]{article}
\usepackage{lmodern}
\usepackage{amssymb,amsmath}
\usepackage{ifxetex,ifluatex}
\usepackage{fixltx2e} % provides \textsubscript
\ifnum 0\ifxetex 1\fi\ifluatex 1\fi=0 % if pdftex
  \usepackage[T1]{fontenc}
  \usepackage[utf8]{inputenc}
\else % if luatex or xelatex
  \ifxetex
    \usepackage{mathspec}
  \else
    \usepackage{fontspec}
  \fi
  \defaultfontfeatures{Ligatures=TeX,Scale=MatchLowercase}
\fi
% use upquote if available, for straight quotes in verbatim environments
\IfFileExists{upquote.sty}{\usepackage{upquote}}{}
% use microtype if available
\IfFileExists{microtype.sty}{%
\usepackage{microtype}
\UseMicrotypeSet[protrusion]{basicmath} % disable protrusion for tt fonts
}{}
\usepackage[left = 3 cm, right = 2 cm, top = 3 cm, bottom = 2 cm]{geometry}
\usepackage{hyperref}
\hypersetup{unicode=true,
            pdfborder={0 0 0},
            breaklinks=true}
\urlstyle{same}  % don't use monospace font for urls
\usepackage{longtable,booktabs}
\usepackage{graphicx,grffile}
\makeatletter
\def\maxwidth{\ifdim\Gin@nat@width>\linewidth\linewidth\else\Gin@nat@width\fi}
\def\maxheight{\ifdim\Gin@nat@height>\textheight\textheight\else\Gin@nat@height\fi}
\makeatother
% Scale images if necessary, so that they will not overflow the page
% margins by default, and it is still possible to overwrite the defaults
% using explicit options in \includegraphics[width, height, ...]{}
\setkeys{Gin}{width=\maxwidth,height=\maxheight,keepaspectratio}
\setlength{\emergencystretch}{3em}  % prevent overfull lines
\providecommand{\tightlist}{%
  \setlength{\itemsep}{0pt}\setlength{\parskip}{0pt}}
\setcounter{secnumdepth}{0}
% Redefines (sub)paragraphs to behave more like sections
\ifx\paragraph\undefined\else
\let\oldparagraph\paragraph
\renewcommand{\paragraph}[1]{\oldparagraph{#1}\mbox{}}
\fi
\ifx\subparagraph\undefined\else
\let\oldsubparagraph\subparagraph
\renewcommand{\subparagraph}[1]{\oldsubparagraph{#1}\mbox{}}
\fi

%%% Use protect on footnotes to avoid problems with footnotes in titles
\let\rmarkdownfootnote\footnote%
\def\footnote{\protect\rmarkdownfootnote}

%%% Change title format to be more compact
\usepackage{titling}

% Create subtitle command for use in maketitle
\providecommand{\subtitle}[1]{
  \posttitle{
    \begin{center}\large#1\end{center}
    }
}

\setlength{\droptitle}{-2em}

  \title{}
    \pretitle{\vspace{\droptitle}}
  \posttitle{}
    \author{}
    \preauthor{}\postauthor{}
    \date{}
    \predate{}\postdate{}
  
\usepackage{geometry}
\usepackage{fontspec}

\begin{document}

\fontspec{Arial}

O objetivo da pesquisa é obter a intenção de votos para a Chapa 1 e
Chapa 2, da eleição de Colegiado do Instituto de Matemática e
Estatística da Universidade Federal Fluminense. Queremos, então, estimar
a proporção de eleitores em cada chapa. Portanto, foi necessário coletar
uma amostra que representasse bem a população de interesse, a fim de se
obter uma boa estimativa. Para coletar a amostra foi utilizado o plano
amostral probabilístico de Amostra Aleatória Simples Sem Reposição,
porque, além da população ser pequena (156 docentes), o prazo não era
tão longo para aplicação de planos amostrais mais elaborados.

Para aplicação do Plano de Amostragem Aleatória Simples Sem Reposição é
necessário possuir um cadastro prévio de todos os elementos da população
alvo, onde cada unidade elementar tem a mesma probabilidade de ser
incluída na amostra. Para sortear a amostra utilizaremos o algoritmo de
Hajek. Uma vez sorteada a amostra, os elementos devem ser entrevistados
a fim de coletar as informações desejadas.

Para começarmos a pesquisa, precisamos calcular o tamanho da amostra.
Para tanto, precisamos adotar uma margem de erro máxima e um nível de
confiança. A margem de erro é um número que estima o maior erro dos
resultados da pesquisa, com base na amostra. Por exemplo, se a margem de
erro de uma pesquisa é de 7\%, e o resultado com base na amostra for de
60\%, devemos considerar que este número na verdade pode oscilar de 53\%
a 67\%. O ideal é que não tenhamos uma margem de erro tão grande, senão
a pesquisa perde precisão. Entretanto, para diminuir a margem de erro é
necessário aumentar o tamanho da amostra. Já o nível de confiança
representa a probabilidade de uma pesquisa ter os mesmos resultados, com
amostras diferentes; desde que a amostra seja proveniente da mesma
população e com mesma margem de erro. Por exemplo, se uma pesquisa é
realizada com nível de confiança de 95\%, isso significa que, se a
pesquisa for feita 100 vezes, em 95 delas o resultado estaria dentro da
margem de erro.

De início, pensamos em utilizar o cenário mais conservador (supor que o
parâmetro de interesse é 0,5), para não precisarmos coletar uma amostra
piloto. Porém, com o cenário conservador, o tamanho da amostra seria de
74 professores, utilizando erro de 7\% e nível de confiança de 90\%.
Para tentar diminuir a margem de erro e/ou aumentar o nível de
confiança, foi requisitada uma amostra piloto, a fim de entrarmos num
cenário mais ``real''.

Uma vez coletada a amostra piloto, obtemos 52,6\% dos votos para a Chapa
1 e 47,4\% para a Chapa 2. Assim, foi possível reduzir a margem de erro
para 6,5\%, mantendo o nível de confiança em 90\% e obtendo uma amostra
de 74 docentes para estimação. O tamanho da amostra foi calculado da
seguinte forma:

\[n = \frac{N}{ \frac{(N-1)D}{p(1-p)} +1 },\]

\noindent onde \(D = \frac{B^2}{z^2_{ \frac{\alpha}{2} }}\) e \(B\) é o
erro máximo.

Utilizando o algoritmo de Hajek, foi criada a lista dos docentes que
deveriam ser entrevistados. Após as informações serem coletadas, foram
realizadas as devidas análises e os resultados estão descritos na Tabela
1.

\newpage

\begin{center}
Tabela 1: Estimativas pontuais e intervalares de ambas as Chapas.
\end{center}

\begin{longtable}[]{@{}lll@{}}
\toprule
& Chapa 1 & Chapa 2\tabularnewline
\midrule
\endhead
Estimativa
Pontual\footnote{Estimativa pontual: Valor da característica de interesse, obtido na amostra.}
& 56,76\% & 43,24\%\tabularnewline
Estimativa
Intervalar\footnote{Estimativa intervalar: Intervalo aleatório que contém o verdadeiro valor do parâmetro}
& {[}50,29\% ; 63,23\%{]} & {[}36,77\% ; 49,71\%{]}\tabularnewline
\bottomrule
\end{longtable}

Portanto, com nível de confiança de 90\%, a Chapa 1 vencerá a eleição de
Colegiado.


\end{document}
